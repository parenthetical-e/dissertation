In fMRI (as in time-series signal analysis in general) there is an intrinsic trade-off between simply detecting that a signal has occurred in the presence of noise and then estimating the timecourse (i.e. shape) of that signal \cite{Dale:1999p7901,Birn:2002p1777,Liu:2004p2141}.   The state of the art method for setting trial ordering in an attempt to maximize both signal detection and estimation is a genetic algorithm design by \citeNP{Kao:2009p7899}.  Without extensive modification unfortunately their methodology cannot account for epoch-style designs\footnote{This flaw is common to all methods of stimulus timing optimization, so far as I am aware.} that this experiment requires; each trial needs to contain stimulus-response, jitter and feedback delivery periods.  To account for this, Koa's methods will be applied twice, once using a 6.5 second ISI (the estimated average length of a trial) and once with a 1 second ISI (the length of the feedback display in task 2).  These two designs will then be manually interpolated to create a series of stimulus-response, jitter, feedback, and ITI events.  To create the final trial ordering each stimulus-response event will then be subsequently randomly recoded to match one of the 6 ``tree'' stimuli.  This randomization step will be performed independently for each subject so that stimulus assignment to events will be counterbalanced across subjects to control for specific item effects.  


