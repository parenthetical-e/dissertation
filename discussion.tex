TODO - rational for causal reasoning between neighbor models; the correction do not matter.  Noise is constant, and we know precisely the cause of any differences

II is classic catergory, yet people used them at rewards with no apperent accuracy penalty.

Our prototype category, inapcting the RPE is (nearly) a better predicter of BOLD changes.

Simply stimuli generalize well (by inference, no training is needed) in humans and animals.  Mechanisticall how this occurs has not been studies.  I argue that the even Simple stimuli like tones have categorical representations and that these representations, reflected in the dopamenergic prediction errors, facilate stimulus generalization.  Note that the simplest animal generalize on the first new (related) example.  This means that categorical representation must be in place prior to that first event.  That is, categories are a native representation of the stimuli, not one learned after the fact.  And indeed a categorical basis for even simple stimuli is advantagoues; due to intrinsic noise in neuronal encoding the same stimulus viewed twice must have a different representation.  A categorical representation could (or should in any case) nativly overcome such noise. 

Simple rewards decay by exp/or guass too.  

Generalizable representations, i.e. categories, are a basic feature.  This basic feature extended to rewards.  This is not to say specfics do not matter - O'Reilly's trade-off paper
