\documentclass[doc,12pt]{apa}        % use: 'man' for submission type; 'jou' for
                                % journal type, and 'doc' for typical latex
                                % but with figures inline with text
\usepackage{geometry} 
%\geometry{a4paper} 
\usepackage[parfill]{parskip}   % paragraphs delimited by an empty line

\usepackage{graphicx} 
\usepackage{amssymb}            % no idea what this does...
\usepackage{epstopdf}           % no idea what this does...
%\usepackage{gensymb}            % no idea what this does...

\usepackage{setspace}

\DeclareGraphicsRule{.tif}{png}{.png}{`convert #1 `dirname #1`/`basename #1 .tif`.png} \setcounter{secnumdepth}{0}  % no idea what this does...

\usepackage{apacite}
%%%%%%%%% END HEADER %%%%%%%%%

\title{Categories of rewards.} 
\author{Erik J. Peterson} \affiliation{Dept. of Psychology \\ Colorado State University \\ Fort Collins, CO} 

%%%%%%%%%%%%%%%%
\begin{document} 
%%%%%%%%%%%%%%%%
\maketitle
\doublespacing

\section{Introduction} % (fold)
\label{sec:introduction}
Birds will peck repeatedly, as mice will push levers, monkeys will hit buttons, and men will buy flowers, if each of these actions is followed by a primary reward -- food and sex.  Buttons, levers and flowers have no value alone, so reinforcement theory goes, it is only by the \emph{statiscally regular pairing} with primary rewards that value is transfered.  This is, what I'll call, the classical view.  And while it is an inadaquate theory in some other regards I'll not discuss (REFS), it has in general held up many years now.  In fact, the neural basis of such learning currently receives much attention, notable progress is bieng made.  However, reinforcement learning theories can't account for two new key findings in the nueral correlates of human learning. (1) Rewards seem to neurally appear by cognition alone. (2) Value can be transfered by inference, no pairing is needed.  For the first time using a mixture of fMRI and computational modeling, I examine possible united mechanisms of both of these aspects. By treating rewards as a kind of category I also offer a framework for extending formal theories of human reinforcement learning to some cognitive and inferential cases.

This introduction has six parts.  First I discuss classical rewards and their neural correlates.  Second I make a case for cognitive rewards, discussing as an example novelty, then moving onto other examples, and finally arguing for the necessity of generalizable reward representations.  Third is a discussion of prior studies of reward generalization in pigeons and other non-human animals as well as in humans, though the literature on the latter is sparse.   All of the above will be done under the banner of the reward prediction error hypothesis of phasic dopamine function (``RPE hypothesis'' from here on).  Which brings us to the fourth section, a diversion discussing alternative non-rewarding theories of phasic dopamine.  Fifth, I breifly review formal models of categorization, which leads into the sixth and final section, the specfic goals and methods of this work.

\newpage
\bibliography {bibmin}

%%%%%%%%%%%%%
\end{document}
%%%%%%%%%%%%%
