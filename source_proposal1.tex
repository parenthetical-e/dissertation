\documentclass[doc]{apa}        % use: 'man' for submission type; 'jou' for
                                % journal type, and 'doc' for typical latex
                                % but with figures inline with text
\usepackage{geometry} 
\geometry{a4paper} 
\usepackage[parfill]{parskip}   % paragraphs delimited by an empty line

\usepackage{graphicx} 
\usepackage{amssymb}            % no idea what this does...
\usepackage{epstopdf}           % no idea what this does...
% \usepackage{gensymb}            % no idea what this does...

\DeclareGraphicsRule{.tif}{png}{.png}{`convert #1 `dirname #1`/`basename #1 .tif`.png} \setcounter{secnumdepth}{0}  % no idea what this does...

\usepackage{apacite}

\usepackage{setspace}
%\onehalfspacing
\doublespacing      

%%%%%%%%% END HEADER %%%%%%%%%

\title{Is novelty a reward?} 
\author{Erik J. Peterson} \affiliation{Dissertation Proposal\\ Fall 2010 \\ Dept. of Psychology \\ Colorado State University \\ Fort Collins, CO} 

%%%%%%%%%%%%%%%%
\begin{document} 
%%%%%%%%%%%%%%%%
\maketitle
\newpage

\section{Introduction} % (fold)
\label{sec:introduction}

Said simply, is novelty a reward?  Classically rewards have been defined as either primary (food and sex) or secondary (events experientially linked to primary rewards).  Both primary and secondary rewards lead to activation of the putative reward circuitry: that is they result in phasic activity in VTA/SNc (ventral tagmental area/substantia nigra pars compacta).  Novelty, salience, goal achievement without explicit feedback and sequence initiation and termination also activate reward circuitry via phasic activity in the VTA/SNc \cite{Wittmann:2008p541,Jin:2010p7199,Tricomi:2008p6663,Roesch:2010p7179,Zink:2003p5107}. As a first step, this proposal focuses on novelty. The question is whether reward-related and novelty-related phasic activity are functionally equivalent;  can novelty-related activity act as a primary reward?    Specifically this proposal seeks to evaluate possible additive interactions between feedback and novelty based on the reward prediction error (RPE) hypothesis of phasic dopamine activity.  This introduction first seeks to explain the empirical and theoretical bases of the RPE hypothesis.  This is followed by a discussion of how phasic dopamine and the basal ganglia drive stimulus-response learning.  Next is a discussion of how novelty also elicits phasic dopaminergic activity and how this has been incorporated into the RPE model. It concludes with the proposed design, which will primarily detect both behavioral effects and BOLD (blood oxygen level dependent) signal changes due to positive and negative feedback interacting with novelty.  Supportive preliminary results are also presented.

\subsection{Phasic dopamine and the reward prediction error hypothesis} % (fold)
\label{sub:phasic_dopamine_and_the_reward_prediction_error_hypothesis}

The VTA/SNc is a small brainstem nucleus whose dopamine-releasing neurons project strongly to both the striatum and the hippocampus.  Electrophysiological recordings of VTA/SNc neurons show two firing modes -- tonic and phasic \cite{DawNW:2006p6343}.  The phasic mode is of interest here.  It has long been known that phasic firing in VTA/SNc immediately follows reward delivery \cite{iversen:2007aa}. \citeNP{Mirenowicz:1994p7185}, observed that the magnitude of phasic activity was dependent not on the absolute or relative value of a reward, as was previously thought, but instead was related to both the value of the reward and how expected that reward was.  This dependence on expectation and outcome was similar, the authors noted, to the reward prediction error (RPE) signal generated in reinforcement learning models developed by researchers in machine learning.  

One of the simplest reinforcement learning models, temporal difference (TD), proceeds as follows: At time $t+1$ an agent selects an action prompted by a familiar stimulus, denoted here as an S-R (stimulus-response) pair or as $v(s,a,t+1)$. Action selection is followed by reward $r(t+1)$, confined to values of either 1 or 0, present or absent.  This reward is compared to the current estimate of value for the S-R pair as well the estimate from the previous time step $v(s,a,t)$.  This comparison is the reward prediction error (RPE) seen in \emph{Eq 1}.   If the reward is greater than expected a positive RPE results.  If the reward is less than expected, a negative RPE is generated.  If expectations are perfectly met the RPE signal is zero.  The RPE is then used by the agent to update the value of the S-R pair, increasing or decreasing its value, respectively, if the RPE was positive or negative (\emph{Eq. 2}).  If the RPE was zero the value of the S-R pair is unchanged.  The rate at which S-R pair value changes depends on one free parameter, the learning rate ($\alpha$).  Just as the RPE signal is necessary for a computational agent to improve its performance, so too, the theory goes, is the phasic firing of dopamine neurons necessary for an animal to learn from reward.

\begin{equation} RPE = r(t+1) + v(s,a,t+1) - v(s,a,t) 	\end{equation}
\begin{equation} v(s,a,t) \leftarrow v(s,a,t) + \alpha*RPE \end{equation}

\citeNP{Fiorillo:2003p6375}, along with \citeNP{Bayer:2005ul}, quantified the relationship between the unexpectedness of a reward and phasic activity in dopamine neurons.  Both groups showed that the RPE from their reinforcement learning models was strongly correlated to the observed dopamine response.  Complementing these electrophysiological recordings of monkey VTA/SNc, fMRI experiments in humans (for example, \citeNP{ODoherty:2003p6329}), as well as recordings in rat \cite{Roesch:2007p2519}, have also found the characteristic patterns of the RPE hypothesis: a phasic increase with unexpected rewards and a phasic depression when expected rewards were omitted.  These latter studies have also showed another important consistency between the RPE signal and phasic dopamine activity -- back-propagation.  For example, if an image of a green arrow often precedes reward delivery, recordings in VTA/SNc will initially show phasic activity immediately after reward delivery.  However, trial after trial this reward-related response will decrease, while simultaneously a phasic response to the green arrow will develop \cite{Roesch:2007p2519}.  That is, the reward response back-propagates to an informative stimulus.   
% subsection phasic_dopamine_and_the_reward_prediction_error_hypothesis (end)

\subsection{Physiological basis of VTA/SNc phasic firing} % (fold)
\label{sub:physiological_basis_of_vta_snc_phasic_firing}
VTA/SNc receives input from the internal globus pallidus or GPi (a major output structure of the basal ganglia with widespread cortical connections), thalamus, and the central nucleus of the amygdala \cite{Botvinick:2008p6594}.  Recent work though has highlighted the habenula (a small nucleus posterior to the thalamus) as being especially important in generating RPE-like phasic activity.  The lateral habenula has reciprocal connections to the GPi and projections into the VTA/SNc. Based on this anatomy, it has been suggested that this nucleus serves a point of intersection between the striatum and the limbic system (e.g. the amygdala, hippocampus and the serotonergic dorsal raphe nucleus) \cite{Hikosaka:2008p4455}.  The habenula acts to tonically inhibit or disinhibit dopamine release in VTA/SNc neurons.  As habenula activity decreases, burst firing in VTA/SNc results; as habenula firing increases VTA/SNc firing is temporarily paused.  Dual recordings of the habenula and GPi suggest they form a functional loop capable of calculating the value of S-R pairs \cite{BrombergMartin:2010p7221}.  Reversible chemical inhibition of the habenula also increases VTA/SNc phasic activity.  Lesions to the habenula also result in marked increases in dopamine levels in the dorsal and ventral striatum \cite{BrombergMartin:2010p7221}.  In summary, the GPi (withs its access to cortical inputs via the striatum) and habenula (with its capability for altering VTA/SNc activity) may form the physiological loop necessary to calculate the RPE. Though the precise origins of the terms of \emph{Eq 1} and \emph{2}. remain unclear, \citeNP{BrombergMartin:2010p7218}, hint that this loop can signal both initial value estimates ($v(s,a,t)$, \emph{Eq 2}.) and rewarding outcomes ($r(t+1)$, \emph{Eq 1}.).
% subsection physiological_basis_of_vta_snc_phasic_firing (end)

\subsection{Dopamine and the striatum} % (fold)
\label{sub:dopamine_and_the_basal_ganglia}
The striatum is an input area of the basal ganglia, a brain region highly involved in categorization, logical inference, habit formation, working memory and feedback mediated S-R learning \cite{Frank:2001p1996,Jin:2010p7199,SchmitzerTorbert:2004p5410,Seger:2008p6401,Seger:2010p7189,Yin:2006p5080}.  In S-R learning, two of the five striatal subregions (the head of the caudate and the ventral striatum) process reward information \cite{Yin:2005p5101,Yin:2008p6347,Schonberg:2009p6669}.  These two are highly innervated by projections from the VTA/SNc, but only the ventral striatum correlates with the RPE signal \cite{Haruno:2006p3979,Seger:2010p7189}.  The remaining three regions (the body and tail of the caudate and the putamen) are involved with S-R pair formation, visual categorization and response selection, respectively \cite{Seger:2008p6401,Seger:2010p7189}.  Though these three also receive VTA/SNc projections and are sometimes sensitive to reward level \cite{BischoffGrethe:2009p4570}, the BOLD signal does not correlate with the RPE \cite{Seger:2010p7189}; dopamine's exact role in these areas is less clear.  Overall though, intact dopamine projections and complete striatal function is necessary for rapid S-R learning.
   
Administering dopamine antagonists to human and non-human animals adversely affects S-R learning \cite{Pizzagalli:2010p7205}, as does lesioning the VTA/SNc.  Complete lesions of the striatum also prevent S-R learning \cite{Packard:2002p5074}.  Administering dopamine agonists or the readily converted precursor L-DOPA leads to increases in response vigor and the ability of a Pavlovian-conditioned stimulus to bias unrelated instrumental responses (i.e. pavlovian instrumental transfer or PIT) \cite{Winterbauer:2007p6352}. Both PIT and response vigor are, in part, facilitated by phasic dopamine increasing activity in the ventral striatum.  Unmedicated Parkinson's patients, who have low striatal dopamine levels, show marked decreases in S-R learning with rewarding outcomes when compared to patients on medication and healthy age and intellect matched controls \cite{Pizzagalli:2010p7205}.  These same patients show an enhanced capability to learn from negative feedback which suggests that decreases in dopamine convey negative outcome information \cite{Frank:2004p4709}.  Finally, there is a solid body of evidence suggesting that phasic dopamine alters the plasticity of neurons in the striatum which presumably facilitates stimulus-response learning \cite{Calabresi:2007p4284}.
% subsection dopamine_and_the_basal_ganglia (end)

\subsection{Dopamine and novelty} % (fold)
\label{sub:dopamine_and_novelty}

The RPE hypothesis is known to be an incomplete account of phasic dopamine activity.  Multiple studies have demonstrated that phasic firing in the VTA/SNc also occurs in response to novel visual and auditory stimulation \cite{Zink:2004p5108,Reed:1996p7250,Blatter:2006p6372}. However less experimental effort, compared to reward studies, has been directed towards understanding this novelty response.  As a result, the exact dynamics of the novelty signal are unclear: how it fast it decays with repeated exposure, whether its omission leads to a pause in dopamine activity similar to reward, and whether it back-propagates to an informative cue.  

Novelty driven phasic dopamine is thought to be driven by partially separate brain regions than those that mediate the reward-related response \cite{Joel:2002p6593,Lisman:2005p5455}.  The hippocampus is thought to play a key role \cite{Lisman:2005p5455}.   The hippocampus is both a source and target of VTA/SNc phasic firing.  The hippocampus/VTA loop begins with the hippocampus detecting novelty, and passing the signal through the subiculum, to ventral striatum, and ventral pallidum to the VTA/SNc which then projects back to the hippocampus.   The dopamine projection from VTA/SNC to the hippocampus facilitates encoding thus increasing the likelihood a memory can be recalled.   It is unclear what portion of dopamine signal that follows from the initial detection of novelty in the hippocampus is calculated in VTA/SNc, the ventral striatum, or the hippocampus and what if any role the habenula plays.  So although hippocampally driven phasic dopamine projects to the striatum, just like reward related phasic dopamine, it is possible that dopamine acts in a capacity outside of the RPE hypothesis in this context.  Fully addressing this question would require repeating many of the same experiments that support the RPE hypothesis: detailed recording from both dopamine and hippocampal neurons while novel stimulation is applied, repeated, and then omitted.  However, positive results in this proposed study would strongly support the notion that hippocampal/novelty driven firing in the VTA/SNc is functionally identical to reward driven activity.

The RPE hypothesis has been successful in explaining the reward related firing of dopaminergic cells in the VTA/SNc and is also a good predictor of behavioral outcomes \cite{Doya:2008p6554,Schonberg:2007p518,Cohen:2007p4218}.  It is tempting then to try and incorporate novelty-driven phasic firing into the RPE hypothesis.  The idea that novelty can act \emph{similar to} reward has been around for sometime though the idea that novelty and reward are reflections of the \emph{same phenomenon} is new.  The novelty bonus -- treating a new stimuli or context as something intrinsically rewarding, something to be sought -- was introduced by \citeNP{Sutton:1990p7239}, as means to sub-optimally solve a separate problem in reinforcement learning, the exploration/exploitation trade off\footnote{There is no optimal solution \cite{Dayan:1996p7238}.}.  More recently, Sutton's idea of the novelty bonus was theoretically extended beyond its initially proposed role in encouraging exploratory behaviors.  In this new theory it was suggested that novelty be treated as if it were a primary reward \cite{Kakade:2002p6414}.

\begin{equation} r(t) \rightarrow r(t+1) + n(u(t+1),T) 	\end{equation}	
\begin{equation} RPE = r(t+1)+ n(u(t),t+1) + v(s,a,t+1) - v(s,a,t) 	\end{equation}

\citeNP{Kakade:2002p6414} purposed that novelty bonuses are well expressed by equations \emph{Eq 3} and \emph{4}.  Compared to TD, the reward term in these equation is transformed via the addition of a term representing the degree of novelty $n(u(t),t+1)$ at the current state ($u(t+1)$, \emph{Eq 3}).  However $r(t+1)$ and $n(u(t),t+1)$ take on the same values, either 1 (present) or 0 (absent).  With this formulation the equation for the RPE becomes \emph{Eq 4}.  Just like when reward is repeatedly experienced, continued exposure to novel stimuli will lead the RPE to zero.  The rate of decreases is again dependent on the learning rate ($\alpha$, see \emph{Eq 2}).  As the novelty bonus is calculated following initial exposure, behavioral effects might be expected only on the second exposure, assuming each exposure exists as a single state.  However behavioral consequences could appear quite rapidly if we assume instead that states are continually updated, moving in lock-step with continued visual inspection (consistent with the experimental work of \citeNP{Nomoto:2010p7209}).
% subsection dopamine_and_novelty (end)

\subsection{Proposed study} % (fold)
\label{sub:proposed_study}

There is compelling initial data consistent with the notion that the RPE hypothesis can be expanded to account for novelty and, as such, reward and novelty interact in an additive fashion.  In an fMRI experiment \citeNP{GuitartMasip:2010p7227}, showed precisely that.  They reported that novel images preceding rewarding outcomes lead to enhanced ventral striatal activity compared to reward alone.  Model-based analyses are similarly suggestive. In a classic exploration/exploitation armed-bandit task, employing both novel and familiar images as cues \citeNP{Wittmann:2008p541}, compared two TD models one with novelty bonuses and one without.  The RPE signal for both models significantly correlated with BOLD signal changes in the ventral striatum, as has been seen previously \cite{Seger:2010p7188,ODoherty:2003p6329,Haruno:2006p3979}.  However, the model containing novelty bonuses was a significantly better fit to the BOLD data.  Both of these results, while compelling initial evidence, only demonstrate that neuronal changes associated with reward and novelty are additive, consistent with \citeNP{Kakade:2002p6414}.  However no novelty/feedback behavioral effects were reported.  An important, even key, feature of the RPE hypothesis is its capacity to predict behavioral performance.

Another feature of the novelty modified RPE hypothesis is that reward and novelty are interchangeable.  Rewarding events that precede a visual stimulus or word-pair lead to enhanced encoding as measured by later recall performance \cite{Lisman:2005p5455}.  Building on that, \citeNP{Wittmann:2007p3328}, showed enhanced recall of images (natural scenes), compared to control, when scenes were preceded by novel images at the time of encoding.  This effect occurred even if the novel image was absent but was anticipated by an informative cue.  On the surface this is similar to effects of reward anticipation observed in studies of striatal function \cite{Knutson:2001p5234}. There is an important caveat to these findings.  Dopamine's action in the hippocampus, to facilitate rapid encoding of long-term visual or semantic memory, is different from its striatal role -- the trial by trial estimating of the value of a stimulus-response pair.  That is, the fact that novelty acts as a reward in the hippocampus does not imply this will also be true in the striatum.

By combining feedback valence with novel image presentation it should be possible to detect separable and additive influences of both feedback and novelty on learning as measured by behavior and neural activity.  This study will employ both traditional feedback and the presentation of novel images in order to manipulate learning in a stimulus-response task.  A ``weather prediction'' cover story will be used;  subjects will learn to associate individual abstract black and white images with one of two responses, ``rain'' and ``sun''.  In such a task learning is normally driven solely by visual feedback, in this case a green (correct) or red (incorrect) background.   In this version, however, every time a participant receives feedback, they will also see a fractal image in the center of the screen.  In one condition (``familiar'') one of three pre-studied fractal images will appear.  In the second condition (``novel/correct'') a novel fractal will appear when they make a correct response, and a familiar image when they make an incorrect response.   In the final condition (``novel/incorrect'') incorrect responses will be accompanied by a novel fractal, while correct responses will be followed with a familiar fractal. I predict, and pilot data shows, that when correct trials are accompanied by novelty, that S-R pair's value increases compared the familiar condition, leading to improved accuracy.  When incorrect trials are accompanied by novelty, the novelty will mitigate the depression in dopamine preventing the normal decrease in that pair's value.  This should decrease the value difference between the two response options, slowing learning compared to the familiar condition. 

The proposed design improves on past work in two important ways. First, it provides for definite behavioral measures (categorization accuracy) following from the additive effects of novelty and reward.  Second, no previous studies have evaluated the potential additive interactions of negative outcomes with novelty.  Establishing the bivalent nature of the dopamine response and its correlation with the bivalent RPE signal was important in establishing the RPE hypothesis, and is an important step to fully determine how novelty relates to reward.
% subsection proposed_study (end)
% section introduction (end)


\section{Proposed methods} % (fold)
\label{sec:methods}
\subsection{Behavioral} % (fold)
\label{sub:behavioral}
The proposed task has two main components, a fractal image familiarization procedure that occurs outside the scanner (for details see Fig. \ref{fig:taskDia}\textbf{A}), and the ``weather prediction'' S-R learning task; Fig. \ref{fig:taskDia}\textbf{B}), As detailed below, subjects will complete this procedure once before the scanning session to ensure that they are capable of learning within this type of task.  Completely different stimuli and fractal images will be used for pre-screening and during fMRI data acquisition.

\begin{figure}[tp]
\fitfigure{novTaskDia} 
\caption{Diagram of the proposed task.  \textbf{A.} In the familiarization procedure, the subject is exposed to three fractal images.  Each image is on screen for 1 second and is seen a total of 40 times.  Presentation order is random.  This procedure will occur immediately prior to the participant entering the scanner for part B.  \textbf{B.} In-scanner task:   Subjects have 2 seconds to respond while the black and white image is on-screen.  After a 0.3 second delay, feedback appears along with a fractal image.  They are three possible feedback conditions, familiar, novel/correct, novel/incorrect (see main text for details). Feedback is displayed onscreen for 0.4 seconds and is then replaced by fixation cross (2-6 sec).}
\label{fig:taskDia}
\end{figure}

Using a modified version of the weather prediction cover story, participants will learn to deterministically categorize 9 black and white abstract images as either ``rain'' or ``sun''.  Each stimulus will be presented 20 times (total trial number: 180, total run time: approximately 14 minutes).   Three stimuli will be assigned to each of the outcome conditions: familiar, novel/correct, or novel/incorrect.  Responses are made (in the scanner) using keypads placed on their uppers thighs or (outside the scanner during pretesting) using two labeled keys on a QWERTY keyboard.  Responses will be made using separate hands with hand assignment counterbalanced across participants.  Following response selection the abstract image is replaced with one of three possible feedback displays based on condition (see Fig. \ref{fig:taskDia}\textbf{B}).  In all three conditions the correctness or incorrectness of the response is indicated by the background of the screen turning green or red, respectively.  In the familiar condition the center of screen displays a random fractal that was studied during the pre-scanning familiarization procedure (see Fig. \ref{fig:taskDia}\textbf{A}).  In the novel/correct condition correct feedback is accompanied by a novel fractal image, while incorrect feedback is accompanied by a familiar fractal.  Conversely, the novel/incorrect condition has incorrect outcomes associated with novel fractals, while correct answers coincide with familiar fractals.   For each participant, the order of the abstract stimuli, outcome conditions, as well as the inter-trial timing, will be optimized for fMRI data acquisition using optSeq2 (http://surfer.nmr.mgh.harvard.edu/optseq/).  Black and white abstract images will be randomly selected for each subject from a pool of 70, as will be the familiar and novel fractals (pool size: 260).  

Prior to beginning both the familiarization procedure and weather prediction task subjects are verbally instructed about the nature of the task.  Key excerpts:

\begin{quote}
Today you are going to be completing a task designed to study trial and error learning.  There are two parts to this task.

Part 1, you will see a set of three fractal images, colorful geometric shapes.  Each image will appear 40 times, the order of appearance is random.  Your job in this part is to memorize each image as well as you are able.  This portion takes less than two minutes.

Part 2 is a task we call weather prediction.  Each trial begins with the appearance of a black and white abstract image.  There are a total of 9 of these images.  Each black and white image is associated with one of two possible responses, RAIN or SUN.  You select a weather category using the two appropriate labeled keys.  After selecting a response, you will be presented with feedback.  If you were right the background of the screen will turn green, if you were wrong it will turn red.

Every time the you receive feedback -- the background turning green (correct) or red (incorrect) -- in the center of the screen will be a fractal image.  Sometimes is will be one of the ones you studied at the beginning of the experiment, sometimes it will be a new image.

So then the goal is to mentally classify each of these fractal images as old or new.  However this must be done while also paying attention to the feedback.  You have to keep track of both.

Finally, you have a limited time window to respond while the black and white image is up.  Additionally, feedback and fractal presentation is rapid so you must pay close attention.
\end{quote}

At no time prior to completing the experiments are participants told that novel fractal image presentation is anything but random.  In pilot studies, during informal interviews post-task, no one reported detecting a pattern in the appearance of novel images.

Pilot studies (next section) showed that between 50--70\% of participants are able to learn the task.  Learning was defined as achieving greater than 70\% accuracy on the last 10 trials.  To exclude non-learners potential participants will be prescreened prior to fMRI scanning.  Prescreening will likely employ a similar two-response design using kanji characters as the stimuli and ``yellow fish'' or ``blue fish'' as the response labels.  No fractal images will be displayed during feedback and the timing of the prescreening task will be identical to the in-scanner task. Verification that this pre-screening is effective in predicting later performance is underway.
% subsection behavoiral (end)


\subsection{fMRI} % (fold)
\label{sub:fmri}
Approximately 10-14 subjects (half female), all right handed and 18-30 years old, will undergo fMRI data acquisition.  Subject prescreening excludes those not fluent in English, with any history of psychiatric or neurological conditions, magnet-incompatible implants, claustrophobia, and brain injury.  The criterion for rejecting a subject's fMRI data: if they fail to respond to $>$10\% of stimuli or move more than 3 mm during scanning as this prevents proper functional to anatomical coregistration.

Visual stimuli will be presented and behavioral responses collecting using functions from the psychtoolbox (http://psychtoolbox.org/) and custom MATLAB (R2007a) scripts running on Mac OS X (version 5.6.5).  In scanner stimuli will presented via a back-projection mirror system.

fMRI data collection will be limited to regions shown in prior work to be most important in both novelty/RPE related processing and S-R learning: the VTA/SNc, the hippocampus, the ventral striatum, and the dorsal striatum (including the head and body of the caudate and the putamen).  These regions can be encompassed by a narrow functional acquisition window of approximately 12 1.8 mm thick slices with a $<$2 mm$^2$ in plane resolution.  This scanning protocol will allow for an extremely rapid image acquisition time, or TR (repetition time), of approximately 700 ms.  Typical fMRI experiments acquire a whole brain functional image in 1500 to 2000 ms.  This improved TR will lead to near a 3 fold increase in statistical power.  This increased power may be necessary. \citeNP{GuitartMasip:2010p7227}, who looking when for similar BOLD changes as those anticipated here, found feedback and novelty interactions only when they limited their analyses to just two or three of the strongest voxels in the ventral striatum.

In addition to functional data acquisition, two types of high-resolution anatomical image sets will be acquired.  The first set will employ a spoiled gradient recalled (SPGR) pulse sequence.  This is the typical sequence employed for anatomical imaging.  While acceptable for cortical and most sub-cortical structures SPGr image sets do not allow for accurate visualization of the VTA/SNc \cite{GuitartMasip:2010p7227}.  As such a second set anatomical images will be acquired using the magnetization transfer (MT) pulse sequence \cite{Helms:2009p7275}.

Subject-specific anatomically-based ROIs (regions of interest) will be defined encompassing the hippocampus, ventral striatum, dorsal striatal subregions (head and body of the caudate and putamen) and the VTA/SNc.  Both right and left hemisphere regions will be drawn.  For the VTA/SNc, some authors recommend treating the VTA and SNc as a single ROI while other reports suggests they exhibit distinct phasic responses \cite{GuitartMasip:2010p7227,Matsumoto:2009p7219}, so both a combined ROI and separate ROIs for these structures will be created.  One set of ROIs will be hand drawn based on the individual SPGR anatomical images, the other based on the MT images.  Prior to a group-level analysis, each subject's data will be individually analyzed.

Initial fMRI analyses will be carried out in Brainvoyager (version 2.1.2, Brain Innovation, Maastricht, The Netherlands). Functional and anatomical images will be subjected to the standard preprocessing steps. In brief, rigid body transformation and elastic 12 sub-volume motion correction is followed by slice time normalization, temporal smoothing using a high pass filter of 3 Hz and spatially smoothed with a Gaussian kernel, full width at half maximum of 4.0 mm.  The BOLD data will also be subjected to auto-regressive (AR(1)) pre-whitening, minimizing false positives due to non-white noise \cite{Smith:2007p6568}.  Finally functional to anatomical coregistration will be carried out.  Brainvoyager-based analyses require that functional to anatomical alignments are done by hand.  However manual alignments may be difficult due to the smaller-than-whole-brain functional acquisition window. Fortunately, MT images have the added benefit of improving automated functional to anatomical alignments, allowing the potential alignment challenges to be minimized by employing automated systems, such as automated brainstem co-registration (ABC) \cite{Napadow:2006p6933} and the subcortical alignment procedures provided by SPM8 (http://www.fil.ion.ucl.ac.uk/spm/).  If automated alignments are required, alignments parameters can be extracted and employed inside Brainvoyager simplifying the reinforcement learning model analysis discussed below. 

\subsubsection{Univariate analyses.}
As is typical for fast event-related fMRI designs, a boxcar function representing the duration of each trial (in TRs) for each condition will be convolved with the hemodynamic response function (HRF) and its derivatives.  This ``design matrix'' will then be regressed against the average BOLD time-course for each ROI, resulting in a set of regression constants (i.e. $\beta$ values) for each condition within each ROI.  \emph{A priori} contrasts of interest are discussed in the \emph{possible fMRI results} section below.

\subsubsection{Model-based analyses.}
Two reinforcement learning models will be constructed based on individual subjects' behavioral data.  The first will be a standard TD (\emph{Eq 1}. and \emph{Eq 2}.). The second will be the novelty bonus modified TD (\emph{Eq 3}. and \emph{Eq 4}.).  Both these models feature two free parameters, $\alpha$ (the learning rate) and $\beta$ which controls the slope of the softmax distribution used to estimate subject's choice behavior \cite{Pessiglione:2006p6481,Sutton:1998fk}.  Both $\alpha$ and $\beta$ will be set, on subject-by-subject and model-by-model basis, using log-likelihood optimization procedure \cite{Pessiglione:2006p6481}.  However as subjects based estimates have been observed to be noisy the subject-average $\alpha$ for each model will be employed in final set of simulations \cite{Yacubian:2006p1676}.  

The RPE signal from each model will then be convolved with the canonical HRF producing a BOLD estimate suitable for analyses inside Brainvoyager.  Each model's HRF estimate will be subjected to a 3 Hz high-pass filter to match preprocessing applied to the BOLD data.  The fit of the two models to the BOLD data will then be compared using the statistical procedure outlined in \citeNP{Ashby:2008p7276} -- essentially an \emph{F}-test with a penalty for the number of free parameters.

For statistical testing to be meaningful, a subset of the voxels in the fMRi data must be pre-selected.  Testing the whole brain for a model fit that is known \emph{a priori} to be localized to isolated regions would, at best, unnecessarily reduce power \cite{Ashby:2008p7276}. The question then is how to select voxels.  Two overlapping methods are proposed.  The first is to test only voxels found in the ROIs for the VTA/SNc, the hippocampus and the ventral striatum.   As described in the introduction there is a strong basis in the literature for selecting these ROIs \cite{GuitartMasip:2010p7227,Seger:2010p7188,ODoherty:2003p6329,Haruno:2006p3979}.  Still, anatomical ROIs are relatively crude and the anatomical regions of interest may be functionally heterogenous.  Both these factors may lead to increased noise.   Anticipating this \citeNP{Ashby:2008p7276}, proposed an objective voxel selection procedure: select N (about 100) voxels that were best fit by the univariate model.  However this may be an overly conservative approach.   The voxels best fit by the univariate model are those least likely to display characteristics similar to those seen in reinforcement learning models.  Alternative objective selection procedures that \emph{may} be considered include, selecting voxels that display the largest variance, or those found to be most informative, in the information theory sense, using the searchlight procedure suggested by \citeNP{Kriegeskorte:2006p4533}.  Voxel selection \emph{may} be employed both inside each ROI, and perhaps, for the entire functional dataset to find those voxels who were the best fit irrespective of spatial location.  
% subsection fmri (end)
% section methods (end)

\section{Behavioral preliminary results} % (fold)
\label{sec:preliminary_results}

\begin{figure}[tp]
\fitfigure{prelim_acc} %% You could use {fig1.eps} here but see note below
\caption{Mean response accuracy for all subjects by outcome condition (x-axis; familiar, novel/correct and novel/incorrect) separated by block (\emph{N}=24).  Dark gray bars indicate subjects who learned (\emph{N}=14), defined as having a mean accuracy of greater than 0.70 for the last 10 trials.  Light gray represent non-leaners (\emph{N}=10).  Changing the learning criterion to the less conservative 0.65 does not appreciably effect the these results. All statistical tests were done for learners only using the proportionality test, an inexact test that the probability of successful outcomes are identical -- *\emph{p = 0.09}; **\emph{p = 0.01}; \#\emph{p = 0.21}.  Multiple comparisons were corrected (by block) using the family-wise error rate.}
\label{fig:prelimAcc}
\end{figure}

Preliminary behavioral results (Fig. \ref{fig:prelimAcc}) are promisingly consistent with the predictions of novelty modified RPE.  Compared to the familiar outcome condition the novel/correct condition shows a significant elevation in mean response accuracy during the strongest learning phase (i.e. in block 2 - trials 10-20) while the novel/incorrect condition shows a trend towards diminished learning during this same period.  Given the marginal \emph{p}-values additional experiments would be needed to confirm these results.  Twenty five to thirty learners should provide sufficient power for reliable statistical outcomes.  However given the high cost of scanning, only 10-12 people will be used for fMRI data acquisition.  It is therefore likely that the differences between conditions will not reach statistical significance in the fMRI group.  While this is not ideal, previous fMRI studies of novelty have also had to rely on additional (non-scanned) subjects to demonstrate behavioral significance (for example, \citeNP{Wittmann:2007p3328}).

\begin{figure}[tp]
\fitfigure{novBoth} %% You could use {fig1.eps} here but see note below
\caption{Mean response accuracy for all subjects by outcome condition (x-axis; novBoth, novel/correct and novel/incorrect) separated by block (\emph{N}=17).  Dark gray bars indicate subjects who learned (\emph{N}=6), defined as having a mean accuracy of greater than 0.65 for the last 10 trials.  Light gray represent non-leaners (\emph{N}=11).  Due to the limited number of learners in this study, the learning criterion was lowered to maximize statistical power.  Due to the limited number of fractal images, the number of trials had to be reduced from 30/stimulus to 20.}
\label{fig:novBoth}
\end{figure}

The novelty modified RPE also predicts that behavioral effects should only arise when novelty is asymmetrically applied to either correct or incorrect outcomes.  There should be no effect on learning when novel images accompany both correct and incorrect outcomes, assuming near equal feedback magnitudes as is the case in this design.  That is, the effect of novelty should be effectively canceled out when it is applied symmetrically.  This appears to be the case.  A second study was run where the familiar outcome was replaced with the ``novBoth'' condition, where novel fractals were associated with both positive and negative outcomes.  The chance of a novel image appearing in this condition was 70\%, the same percent experienced, on average, as the novel/correct condition in previous studies.  As seen in Fig \ref{fig:novBoth}, the novel/correct condition led to better learning than the novBoth condition, while accuracy in the novel/incorrect condition was approximately equal to that in novBoth.  The lack of statistical significance is likely due to the low number of learners.  Additional evidence for the ineffectiveness of the novBoth condition comes from comparing its mean accuracy to familiar accuracy across the two pilot studies.  The block 2 means for the familiar (Fig \ref{fig:prelimAcc}) and NovBoth (Fig \ref{fig:novBoth}) conditions were nearly identical at 0.68 and 0.67 respectively.

\section{Additional proposed pilot studies} % (fold)
\label{sub:additional_proposed_pilot_studies}
While fractal images were selected to serve as the novel/familiar stimuli because they are both abstract (thus unlikely to trigger any recognition or recall due to prior exposure) yet visually complex and distinct (helping in-task recall), they \emph{may} be intrinsically attractive to participants.  As attractive faces can activate the reward circuitry \cite{Aharon:2001p7277} it is possible the effects described above are due, in part, to the rewarding properties of the images themselves decoupled from novelty.  Familiar and novel images are randomly assigned on a subject by subject basis, minimizing this potential confound.  Still there are a larger number of fractals (260) compared to the number of subjects to be run (\emph{N}=24 at current, likely rising to 50 before fMRI piloting begins) so it possible random assignment will be insufficient.  As such, the affective properties of fractal pool will be behaviorally assessed based on the guidelines of the International Affective Picture System database (IAPS; http://csea.phhp.ufl.edu/Media.html\#topmedia).  If the the distribution of the affective ratings is not evenly distributed around a neutral rating, the set will be curated such that this is the case and/or alternative abstract images will be selected from the IAPS database.

To ensure that the novelty effects reported above are not dependent on the comparative nature of the current design (i.e. familiar, novel/correct and novel/incorrect inside the same task), a between-subjects version of the task will be run.  Three randomly assigned sets of participants will complete either a version with only familiar outcomes, novel/correct outcomes or novel/incorrect outcomes.  Accuracy across blocks will then be compared as in Fig \ref{fig:prelimAcc}.  While the overall effects should be maintained, the effect sizes may be altered beyond the decrease in statistical power resulting from a between design  There is a limited body of evidence suggesting that the magnitude of phasic firing in the VTA/SNc, and thus the RPE signal, is normalized based on the range of values experienced.  This scaling appeared similar to reward value divided by the variance \cite{Tobler:2005p6373}.  This variance normalization may mitigate the magnitude of the phasic novelty signal thus reducing its behavioral effects.    

A final pilot study will be run (\emph{N}=20), just prior to fMRI data acquisition, perfectly matching the in-scanner version of the task, including the jittered intra-trial timing necessary for rapid event related designs.   In this version subjects will be pre-screened as outlined in the methods section.  

It is also possible that the in-scanner version of the task may be repeated twice for each subject, using different stimuli and fractal images for each run.  While this approach would significantly increase statistical power, there is a concern that the current black and white stimuli might facilitate transfer from the first run of the task to the second.  This possibility as well as the use of alternative stimuli is under investigation.
% subsection additional_proposed_pilot_studies (end)
% section preliminary_results (end)

\section{Discussion of possible fMRI results} % (fold)
\label{sec:discussion_of_possible_fmri_results}
\subsection{Univariate analyses} % (fold)
\label{sub:univariate_analyses}
A consistency check contrast will be performed on all subjects' data prior to the detailed ROI analyses discussed below.  When all three outcome conditions are contrasted to the fixation baseline (``all$>$base'') positive BOLD activation should be observed in all ROIs (left and right hippocampus, ventral striatum, head and body of the caudate, and putamen as well as VTA/SNc) \cite{Seger:2010p7188,GuitartMasip:2010p7227}.  If negative or no activations are observed in this contrast, detailed by-condition (familiar to base, novel/condition to base and novel/incorrect to base) contrasts will be conducted in effort to determine if the aberrant results are due to averaging of the different conditions.  If this is case analyses will proceed as described.  If not, ROI analyses will be limited to only those regions active during the ``all$>$base'' contrast.

The question posed in this project, do novelty and feedback interact in additive fashion?, is directly addressed by comparing positive and negative  outcomes as a function of familiar or novel image presentation.  By employing the familiar condition as a baseline, the stimulus processing and response contributions to the observed BOLD signal should be effectively cancelled out as the task for each condition is identical. Analyses will be limited to subject-specific ROIs.  Statistical effects (ANOVA and post-hoc multiple comparison corrected \emph{t}-tests) across participants will be conducted inside Brainvoyager following by-subject analyses. Previous work in the lab both in the striatum and the hippocampus suggests that any observed effects will occur bilaterally.  This discussion assumes the current proposal will follow that trend.  To be conservative left and right ROIs will be examined  separately at first, and then combined if no hemispheric differences are found.  The predictions for each ROI are for block 2, the block in which significant behavioral results are observed.  Identical contrasts as those outlined below will also be employed when analyzing block 1.  Results in this block though are difficult to anticipate.  It is possible that neural activity may precede behavioral, in which case BOLD signal changes in block 1 should be similar to block 2.

To prevent confusion the following nomenclature will be employed to describe univariate contrasts.  The three outcome conditions will be referred to as they have been previously - familiar, novel/correct and novel/incorrect.  However each will be modified by either a $(+)$ or $(-)$ which indicate whether correct or incorrect trials are included.  For example, familiar$^{(+)}$ indicates trials in the familiar condition in which the participant was correct and received positive feedback.  Novel/incorrect$^{(-)}$ refers to trials in novel/incorrect condition where subjects were incorrect and received negative feedback.  The lack of either modifier then refers to all trials, both correct and incorrect.
%H
In the VTA/SNc a contrast of novel/correct$^{(+)}$ with familiar$^{(+)}$ should produce a significant positive BOLD response.  The dopaminergic contribution to the BOLD signal should be larger when novelty combines with reward when compared to reward alone.  For the same reason the novel/incorrect$^{(-)}$ with familiar$^{(-)}$ contrast should lead to a significant effect, however the direction of the BOLD signal is uncertain.  It may be that the novelty bonus is insufficiently large, on average, to completely cancel out the expected depression following an incorrect outcome.  If this is the case, the BOLD response will be negative.  If the bonus is large enough to overcome the depression, an positive BOLD signal will result.  When novel/correct$^{(-)}$ is compared to familiar$^{(-)}$ and novel/incorrect$^{(+)}$ to familiar$^{(+)}$ no significant effects are expected.  The dopaminergic contribution should be equal; novelty is not present and feedback valence is identical.  However when novel/correct$^{(-)}$ is compared to familiar$^{(+)}$ a significant decrease in the BOLD signal is expected.  When novelty is held constant, negative outcomes should have a decreased dopaminergic BOLD contribution, compared to positive outcomes.  Inversely, novel/incorrect$^{(+)}$ contrasted to familiar$^{(-)}$ should result in a significant positive response.

The hippocampus is known to generate positive BOLD signal changes to both familiar and novel images, though to the latter more strongly \cite{Johnson:2009p6667}.  As such, when contrasts like those in the previous paragraph are employed the BOLD signal should contain both dopaminergic and recall/recognition/novelty-detection contributions. The combined effects of these components result in the following predictions: when novel/correct$^{(+)}$ is contrasted with familiar$^{(+)}$ significant positive BOLD change is expected.  The same is true for the novel/incorrect$^{(-)}$ with familiar$^{(-)}$ contrast.  The increase should be smaller for the former compared to the latter as the novelty bonus from novel/incorrect$^{(-)}$ with familiar$^{(-)}$ contrast is mitigated by depression following negative outcomes.

Two separate contributors to the BOLD signal in the hippocampus may be detectable, although not completely separable with the current design.  The first contribution comes from initial novelty detection and the second from the innervation this region receives from the VTA/SNc, that is the dopaminergic contribution.  The former must precede the latter.  So when activity in the hippocampus is compared to either the ventral striatum or the VTA/SNc, activity should be longer lived.  While this time-course should consist of two impulses (recognition/novelty-detection and the later dopamine burst) it is unclear whether we have sufficient resolution to detect these events.  They may manifest as a single prolonged response.  In either case, the key predictions are that the hippocampal signal will (1) begin before and (2) significantly outlast both the VTA/SNc and the striatum.  To address (1) the peak onset of the VTA/SNc, ventral striatum and hippocampus will be compared for the novel/correct$^{(+)}$ to familiar$^{(+)}$ contrast using the ``lag to peak'' function in Brainvoyager.  This tool allows for a precise quantification of the time difference between BOLD signal peaks among ROIs.  Prediction (2) and to a lesser degree (1) will be addressed using the same contrast to create a reconstruction of the event-related BOLD response of an average trial.  The resulting average trial BOLD response for the hippocampus should be larger in both magnitude and duration than the average trial reconstruction for the VTA/SNc or the ventral striatum.

Neglecting the time-course differences, ventral striatum activity patterns should be very similar to those in the hippocampus.  Unlike the VTA/SNc the ventral striatum often responds positively to both positive and negative outcomes (though see the \emph{Assumptions} section).  As in the hippocampus this will lead the negative BOLD signal predictions made for the VTA/SNc to become positive activations in the ventral striatum.  Unlike the hippocampus though there should be no contribution to the BOLD signal from recall/recognition, which was predicted to reduce effect sizes in the hippocampus.  As such, effect sizes in the ventral striatum may be comparatively larger.  Activity in the head of the caudate should be similar to that in the ventral striatum.  These regions are both strongly implicated in reward processing are and generally exhibit similar BOLD signal patterns.  The head and ventral striatum are functionally differentiated based on the fact that head activation requires a task-relevant action while the ventral striatum does not \cite{ODoherty:2004p1269,Yin:2008p6347}.  That distinction is not relevant to the current design since task-relevant actions are always performed.

The body of the caudate and the putamen are generally not implicated in reward or novelty processing but are thought to play roles in associating visual stimuli to actions and in response selection, respectively \cite{Seger:2008p6401,Seger:2010p7188}.  As such comparing the novelty conditions to the familiar condition as a function of trial accuracy ($(+)$ and $(-)$, as above) is not expected to yield significant BOLD changes.  However by contrasting all trials from the novel/correct condition with the novel/incorrect \emph{direct} reward and novelty effects should be minimized leaving only the BOLD contribution due to value (or $v(s,a,t)$).  Positive BOLD changes are expected in this contrast as both these regions are involved with responding and are therefore the regions most likely to display neural correlates of value, as has been previously shown \cite{Seger:2010p7188,Haruno:2006p3979}. Values should differ across conditions, driven by the novelty modified RPE.  On average, S-R pairs in the novel/correct condition should be more valuable than those in the novel/incorrect.


\subsection{Important assumptions} % (fold)
\label{sub:important_assumptions}
The discussion above assumed that (1) VTA/SNc can be grouped and the activity in this region follows the ``positive'' coding scheme for rewards generally used in studies of dopamine and RPE hypothesis and (2) ventral striatum will respond to both positive and negative outcomes with an increased BOLD signal (compared to fixation baseline). 

In the positive scheme (as discussed in the \emph{Methods-fMRI} section) rewards are coded as 1 and punishments or reward omission as 0.  There is limited electrophysiological evidence suggesting that neurons in the SNc use this code while cells in the VTA employ a ``negative'' code where rewards are 0 and punishments 1.  This scheme leads to negative RPEs for positive outcomes and positive RPEs for negative outcome. While all fMRI experiments to date suggest positive coding and uniformity in VTA/SNc the author is aware of no instance where this has been explicitly confirmed by creating separate ROIs for VTA and SNc.  It is therefore possible that positive-consistent reports were the result of an mis-averaging of the two regions.  If dual schemes are indicated predictions for univariate analyses can be adjusted to compensate without a loss in analytical to statistical power.  An additional set of reinforcement learning models using the negative scheme could also be easily added.  Additionally all novelty related recordings are consistent with a positive scheme; it is therefore unlikely that the adjustments to the additive predictions need to be made even if dual schemes are in the end required to explain other portions of the data.

With regard to assumption two, initial theories of ventral striatal function suggested this region was involved solely in processing positive outcomes, however recent fMRI studies have conclusively shown involvement in processing aversive outcomes as well.  What is not clear is the direction of BOLD signal changes.  Some report increases while others show decreases \cite{Levita:2009p7280,Wrase:2007p5236,Seger:2005pd,Seger:2010p7188}.  While the majority show positive changes, if this assumption is incorrect the relative differences previously discussed for the ventral striatum would remain unchanged while the signs invert.  Similar inversions \emph{might} also follow in hippocampus and head of the caudate.  How neuronal coding schemes are interchanged among the regions of interest is unknown.
% subsection important_assumptions (end)
% subsection univariate_analyses (end)

\subsection{Model-based analyses} % (fold)
\label{sub:model_based_analyses}
% You need a couple sentences here reminding the reader that you have two models and what they are....then start by saying that RPE should correlate with BOLD as in previous studies.  Then you can get into your predictions about the novelty modified RPE.

As described in the \emph{model-based} portion of the methods section, two reinforcement learning models will be created based on individual subject's behavioral data.  The first will be a standard TD model, one without a novelty term (\emph{Eq 1} and \emph{2.}).  The second will include a novelty bonus (\emph{Eq 3} and \emph{4.}).  The fit of these models to the BOLD signal will then be compared, without and without voxel selection \cite{Ashby:2005p4764}, in the ventral striatum, the VTA/SNc and the hippocampus.  Model fits will also be tested, independent of the ROIs, by selecting \emph{N} voxels from the entire fMRI dataset.

If BOLD changes were to play out as described in \emph{univariate} section, it is highly likely that the novelty modified RPE should be a better fit in the VTA/SNc and the ventral striatum, the two regions whose BOLD signals are consistently correlated with the RPE. \cite{Seger:2010p7188,ODoherty:2003p6329,Haruno:2006p3979}.  It also likely that the voxel selected ROI and full ROI tests described in the Methods section should yield similar results.  As for the voxel-selected whole brain comparisons (i.e. testing the \emph{N} best voxels irrespective of spatial location) voxels in the ventral striatum and VTA/SNc should be most responsive.  To what degree the hippocampus will correlate with either model is unclear.  There are no reports of the hippocampal BOLD signal correlating with the RPE.  Though of course the novelty modified RPE may change that. In the unexpected case that the normal RPE is a better fit than the novelty version, other reinforcement learning models containing a novelty bonus will be considered, such as the alternative approach contained in \citeNP{Kakade:2002p6414}.  
% subsection model_based_analyses (end)


% section discussion_of_possible_fmri_results (end)
\newpage
\bibliography{bibMin}



%%%%%%%%%%%%%
\end{document}
%%%%%%%%%%%%%
